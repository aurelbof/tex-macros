\usepackage{tikz} % tikz
\usepackage{standalone} % tikz figures are often in standalone
\usepackage{amsfonts} % mathbb etc
\usepackage{amsmath} % crucial package
\usepackage{amssymb} % symbols such as \nmid
\usepackage{mathtools} % add/corr to asmmath, \vdotswithin
%\usepackage{todonotes} % very useful
\usepackage{algorithm} % for algorithm environment
\usepackage{algpseudocode} % for pseudocode inside algorithm environment. replaces algorithmic
\usepackage{multirow} % for \multirow inside tabular environment
\usepackage{booktabs} % for top/mid/bottomrule
\usepackage{hwemoji} % for emojis

\usepackage{bm} % bold math symbols
\usepackage{xfrac} % prettier inline fractions
\usepackage{xcolor} % colors
\usepackage{xcolor-solarized} % more colors
\usepackage{xspace} % put at the end of command definitions to add spaces
\usepackage[T1]{fontenc} % modern encoding, better than OT1, the default
\usepackage{hyperref} % clickable links + pdf metadata
\usepackage{hyperxmp} % enhances hyperref for xmp compliance
\usepackage{orcidlink} % for clickable orcid links in author
\usepackage{subcaption} % like subfigure, but better
%\usepackage{cleveref} % for clever cross-referencing with \cref
\usepackage{pifont} % checkmark and crossmark

\usepackage{amsthm} % ams-style theorems
\newtheorem{definition-fr}{Définition}
\newtheorem{example-fr}{Exemple}
\newtheorem{theorem-fr}{Théorème}
\newtheorem{proposition-fr}{Proposition}
\newtheorem{lemma-fr}{Lemme}
\newtheorem{remark-fr}{Remarque}
\newtheorem{corollary-fr}{Corollaire}

% some document classes pre-define these, i think
\makeatletter
\@ifundefined{definition}{\newtheorem{definition}{Definition}}{}
\@ifundefined{lemma}{\newtheorem{lemma}{Lemma}}{}
\@ifundefined{proposition}{\newtheorem{proposition}{Proposition}}{}
\@ifundefined{theorem}{\newtheorem{theorem}{Theorem}}{}
\@ifundefined{remark}{\newtheorem{remark}{Remark}}{}
\@ifundefined{corollary}{\newtheorem{corollary}{Corollary}}{}
\makeatother

\usetikzlibrary{decorations.pathmorphing} % for snake, zigzag...
\usetikzlibrary{decorations.pathreplacing} % for braces
\usetikzlibrary{fit} % for fit=(A) (B) (C)...
\usetikzlibrary{positioning} % for the "below of=1cm" type of options
\usetikzlibrary{intersections} % e.g. for pyramid
\usetikzlibrary{calc} % for computing node coordinates from others
\usetikzlibrary{math} % for pgf math stuff
\usetikzlibrary{overlay-beamer-styles} % when \pause glitches in tikz beamer
\usetikzlibrary{arrows} % arrows
\usetikzlibrary{external} % only compile the figure the first time
\tikzexternalize[only named=true, prefix=tikz-cache/]
\immediate\write18{mkdir -p latex-build/tikz-cache}

% used to pass "scale=XX" as includetikz option
\pgfkeys{
  /tikzoptions/.is family, % Namespace
  /tikzoptions/.cd, % Namespace
  scale/.store in=\scale, % Store the value of 'scale'
}

% oplus node, usage \node[oplus=1cm] or any radius value
\tikzset{
  oplus/.style={
    circle, draw,
    minimum size=2*#1, % diameter = 2 * radius
    inner sep=0pt,
    transform shape=true,
    append after command={
      % Draw vertical line
      \pgfextra{
        \def\r{#1}
        % Draw vertical line
        \draw ($(\tikzlastnode) + (0,-\r)$) --
        ($(\tikzlastnode) + (0,\r)$);
        % Draw horizontal line
        \draw ($(\tikzlastnode) + (-\r,0)$) --
        ($(\tikzlastnode) + (\r,0)$);
      }
    },
  }
}

% usage: \includetikz[scale=XX]{fig}, fetches from my local database
\newcommand{\includetikz}[2][scale=1]{
  \bgroup
  \pgfkeys{/tikzoptions,#1} % currently only accepts 'scale' (april 9 2025)
  \tikzsetnextfilename{#2}
  \include{tex-macros/tikz-figures/#2}% Include the standalone TikZ file
  \egroup
}

% usage: \inputtikz[scale=XX]{fig}, fetches from my local database
\newcommand{\inputtikz}[2][scale=1]{
  \bgroup
  \pgfkeys{/tikzoptions,#1} % currently only accepts 'scale' (april 9 2025)
  \tikzsetnextfilename{#2}
  \input{tex-macros/tikz-figures/#2}% Include the standalone TikZ file
  \egroup
}


\newcommand{\bigOt}{\widetilde{\mathcal{O}}}
\newcommand{\bigO}{\mathcal{O}}

\colorlet{my-purple}{solarized-violet}
\colorlet{my-red}{solarized-red}
\colorlet{my-orange}{solarized-orange}
\definecolor{my-darkor1}{HTML}{96240f}
\definecolor{my-darkor2}{HTML}{7f180c}
\definecolor{my-lb1}{HTML}{00557e}
\definecolor{my-lb2}{HTML}{013157}
\colorlet{my-blue}{solarized-blue}
\definecolor{my-bl1}{HTML}{0d08a9}
\definecolor{my-bl2}{HTML}{04037f}
\colorlet{my-cyan}{solarized-cyan}
\colorlet{my-green}{solarized-green}
\definecolor{my-gr1}{HTML}{097e3e}
\definecolor{my-gr2}{HTML}{025520}
\colorlet{my-yellow}{solarized-yellow}
\colorlet{my-magenta}{solarized-magenta}

% https://personal.sron.nl/~pault/
\definecolor{tolblue}{HTML}{4477AA}
\definecolor{tolred}{HTML}{EE6677}
\definecolor{tolgreen}{HTML}{228833}
\definecolor{tolyellow}{HTML}{CCBB44}
\definecolor{tolcyan}{HTML}{66CCEE}
\definecolor{tolpurple}{HTML}{AA3377}
\definecolor{tolgrey}{HTML}{BBBBBB}

% tol muted
\definecolor{tolmgrey}{RGB}{221, 221, 221}
\definecolor{tolmdarkblue}{RGB}{46, 37, 133}
\definecolor{tolmdarkgreen}{RGB}{51, 117, 56}
\definecolor{tolmgreen}{RGB}{93, 168, 153}
\definecolor{tolmblue}{RGB}{148, 203, 236}
\definecolor{tolmyellow}{RGB}{220, 205, 125}
\definecolor{tolmdarkpink}{RGB}{194, 106, 119}
\definecolor{tolmpurple}{RGB}{159, 74, 150}
\definecolor{tolmdarkpurple}{RGB}{126, 41, 84}

%https://www.nceas.ucsb.edu/sites/default/files/2022-06/Colorblind%20Safe%20Color%20Schemes.pdf
% QUALITATIVE PALETTE
\definecolor{oiblack}{RGB}{0, 0, 0}
\definecolor{oigreen}{RGB}{0, 158, 115}
\definecolor{oiblue}{RGB}{0, 114, 178}
\definecolor{oicyan}{RGB}{86, 180, 233}
\definecolor{oiyellow}{RGB}{240, 228, 066}
\definecolor{oiorange}{RGB}{230, 159, 0}
\definecolor{oired}{RGB}{213, 094, 0}
\definecolor{oipurple}{RGB}{204, 121, 167}

\newcommand{\blue}[1]{{\color{oiblue} #1}}
\newcommand{\green}[1]{{\color{oigreen} #1}}
\newcommand{\cyan}[1]{{\color{oicyan} #1}}
\newcommand{\yellow}[1]{{\color{oiyellow} #1}}
\newcommand{\orange}[1]{{\color{oiorange} #1}}
\newcommand{\red}[1]{{\color{oired} #1}}
\newcommand{\purple}[1]{{\color{oipurple} #1}}

% DIVERGING PALETTE
\definecolor{blue4}{RGB}{16, 101, 171}
\definecolor{blue3}{RGB}{58, 147, 195}
\definecolor{blue2}{RGB}{142, 196, 222}
\definecolor{blue1}{RGB}{209, 229, 240}
\definecolor{whitegrey}{RGB}{249, 249, 249}
\definecolor{red1}{RGB}{254, 219, 199}
\definecolor{red2}{RGB}{246, 164, 130}
\definecolor{red3}{RGB}{215, 95, 76}
\definecolor{red4}{RGB}{179, 21, 41}

\setcounter{tocdepth}{2} % for metadata: 0 = to title, 3 = to subsubsecs

\newcommand{\myorcid}{0009-0003-0016-2667}
%\newcommand{\todoAU}[2][]{\todo[color=my-blue!30,#1]{Aurélien: #2}}

\usefonttheme{structurebold} % bold sans serif titles etc
\useinnertheme{rectangles} % itemize et enumerate: rectangles
\useoutertheme[subsection=false]{miniframes} % clickable dots on headline

\setbeamertemplate{headline} % customize miniframes headline
{
  \begin{beamercolorbox}
    [colsep=1.5pt]{upper separation line head}
  \end{beamercolorbox}
  \begin{beamercolorbox}[ht=2.5ex,dp=2.5ex]
    {section in head/foot} % Navigation bar
    \insertnavigation{0.8\paperwidth}
  \end{beamercolorbox}
}

\setbeamertemplate{footline}[frame number]{} % only frame number on footline
\setbeamertemplate{navigation symbols}{} % no navigation symbols
\setbeamertemplate{caption}{\raggedright\insertcaption\par} % no "Figure:"
%\setbeamertemplate{page number in head/foot}[appendixframenumber]

\setbeamertemplate{frametitle} % customize frame title
{
  \begin{centering}
    \insertframetitle\par
  \end{centering}
}

\usepackage{appendixnumberbeamer} % appendix pages don't count

%\usepackage{caption}
%\captionsetup[figure]{labelformat=empty} % no more fig

\newcommand{\cmark}{\textcolor{my-gr1}{\ding{51}}}
\newcommand{\xmark}{\textcolor{my-red}{\ding{55}}}

\usecolortheme[named=my-lb2]{structure}
\setbeamercolor{section in head/foot}{bg=my-lb1!25} % background color of nav

\AtBeginSection[]{
  \begin{frame}
    \tableofcontents[currentsection]
  \end{frame}
}
