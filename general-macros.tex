\usepackage{tikz} % tikz
\usepackage{standalone} % tikz figures are often in standalone
\usepackage{amsfonts} % mathbb etc
\usepackage{amsmath} % crucial package
\usepackage{amsthm} % ams-style theorems
\usepackage{bm} % bold math symbols
\usepackage{xcolor} % colors
\usepackage{xcolor-solarized} % more colors
\usepackage[T1]{fontenc} % modern encoding, better than OT1, the default
\usepackage[pdfusetitle]{hyperref} % clickable links + pdf metadata

\newtheorem{definition-fr}{Définition}
\newtheorem{example-fr}{Exemple}
\newtheorem{theorem-fr}{Théorème}
\newtheorem{proposition-fr}{Proposition}
\newtheorem{lemma-fr}{Lemme}
\newtheorem{remark-fr}{Remarque}

% some document classes pre-define these, i think
\makeatletter
\@ifundefined{definition}{\newtheorem{definition}{Definition}}{}
\@ifundefined{lemma}{\newtheorem{lemma}{Lemma}}{}
\@ifundefined{proposition}{\newtheorem{proposition}{Proposition}}{}
\@ifundefined{remark}{\newtheorem{remark}{Remark}}{}
\makeatother

\usetikzlibrary{decorations.pathmorphing} % for snake, zigzag...
\usetikzlibrary{decorations.pathreplacing} % for braces
\usetikzlibrary{fit} % for fit=(A) (B) (C)...
\usetikzlibrary{positioning} % for the "below of=1cm" type of options
\usetikzlibrary{intersections} % e.g. for pyramid
\usetikzlibrary{calc} % for computing node coordinates from others
\usetikzlibrary{math} % for pgf math stuff
\usetikzlibrary{overlay-beamer-styles} % when \pause glitches in tikz beamer
\usetikzlibrary{arrows} % arrows
\usetikzlibrary{external} % only compile the figure the first time
\tikzexternalize[only named=true, prefix=tikz-cache/]
\immediate\write18{mkdir -p latex-build/tikz-cache}

% used to pass "scale=XX" as includetikz option
\pgfkeys{
  /tikzoptions/.is family, % Namespace
  /tikzoptions/.cd, % Namespace
  scale/.store in=\scale, % Store the value of 'scale'
}

% oplus node, usage \node[oplus=1cm] or any radius value
\tikzset{
  oplus/.style={
    circle, draw,
    minimum size=2*#1, % diameter = 2 * radius
    inner sep=0pt,
    transform shape=true,
    append after command={
      % Draw vertical line
      \pgfextra{
        \def\r{#1}
        % Draw vertical line
        \draw ($(\tikzlastnode) + (0,-\r)$) --
        ($(\tikzlastnode) + (0,\r)$);
        % Draw horizontal line
        \draw ($(\tikzlastnode) + (-\r,0)$) --
        ($(\tikzlastnode) + (\r,0)$);
      }
    },
  }
}

% usage: \includetikz[scale=XX]{fig}, fetches from my local database
\newcommand{\includetikz}[2][scale=1]{
  \bgroup
  \pgfkeys{/tikzoptions,#1} % currently only accepts 'scale' (april 9 2025)
  \tikzsetnextfilename{#2}
  \include{tex-macros/tikz-figures/#2}% Include the standalone TikZ file
  \egroup
}

% usage: \inputtikz[scale=XX]{fig}, fetches from my local database
\newcommand{\inputtikz}[2][scale=1]{
  \bgroup
  \pgfkeys{/tikzoptions,#1} % currently only accepts 'scale' (april 9 2025)
  \tikzsetnextfilename{#2}
  \input{tex-macros/tikz-figures/#2}% Include the standalone TikZ file
  \egroup
}


\newcommand{\bigOt}{\widetilde{\mathcal{O}}}
\newcommand{\bigO}{\mathcal{O}}

\colorlet{my-purple}{solarized-violet}
\colorlet{my-red}{solarized-red}
\colorlet{my-orange}{solarized-orange}
\definecolor{my-darkor1}{HTML}{96240f}
\definecolor{my-darkor2}{HTML}{7f180c}
\definecolor{my-lb1}{HTML}{00557e}
\definecolor{my-lb2}{HTML}{013157}
\colorlet{my-blue}{solarized-blue}
\definecolor{my-bl1}{HTML}{0d08a9}
\definecolor{my-bl2}{HTML}{04037f}
\colorlet{my-cyan}{solarized-cyan}
\colorlet{my-green}{solarized-green}
\definecolor{my-gr1}{HTML}{097e3e}
\definecolor{my-gr2}{HTML}{025520}
\colorlet{my-yellow}{solarized-yellow}
\colorlet{my-magenta}{solarized-magenta}


\hypersetup{
  colorlinks = true,
  linkcolor = my-red,
  citecolor = my-green,
  filecolor = my-cyan,
  urlcolor = my-magenta
}


