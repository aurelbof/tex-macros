\usepackage{tikz} % tikz
\usepackage{standalone} % tikz figures are often in standalone
\usepackage{amsfonts} % mathbb etc
\usepackage{amsmath} % crucial package
\usepackage{amssymb} % symbols such as \nmid
\usepackage{mathtools} % add/corr to asmmath, \vdotswithin
\usepackage{todonotes} % very useful
\usepackage{algorithm} % for algorithm environment
\usepackage{algpseudocode} % for pseudocode inside algorithm environment. replaces algorithmic
\usepackage{multirow} % for \multirow inside tabular environment
\usepackage{booktabs} % for top/mid/bottomrule

\usepackage{bm} % bold math symbols
\usepackage{xfrac} % prettier inline fractions
\usepackage{xcolor} % colors
\usepackage{xcolor-solarized} % more colors
\usepackage{xspace} % put at the end of command definitions to add spaces
\usepackage[T1]{fontenc} % modern encoding, better than OT1, the default
\usepackage[pdfusetitle]{hyperref} % clickable links + pdf metadata
\usepackage{hyperxmp} % enhances hyperref for xmp compliance
\usepackage{orcidlink} % for clickable orcid links in author
\usepackage{subcaption} % like subfigure, but better
\usepackage{cleveref} % for clever cross-referencing with \cref
\usepackage{pifont} % checkmark and crossmark

\usepackage{amsthm} % ams-style theorems
\newtheorem{definition-fr}{Définition}
\newtheorem{example-fr}{Exemple}
\newtheorem{theorem-fr}{Théorème}
\newtheorem{proposition-fr}{Proposition}
\newtheorem{lemma-fr}{Lemme}
\newtheorem{remark-fr}{Remarque}
\newtheorem{corollary-fr}{Corollaire}

% some document classes pre-define these, i think
\makeatletter
\@ifundefined{definition}{\newtheorem{definition}{Definition}}{}
\@ifundefined{lemma}{\newtheorem{lemma}{Lemma}}{}
\@ifundefined{proposition}{\newtheorem{proposition}{Proposition}}{}
\@ifundefined{theorem}{\newtheorem{theorem}{Theorem}}{}
\@ifundefined{remark}{\newtheorem{remark}{Remark}}{}
\@ifundefined{corollary}{\newtheorem{corollary}{Corollary}}{}
\makeatother

\usetikzlibrary{decorations.pathmorphing} % for snake, zigzag...
\usetikzlibrary{decorations.pathreplacing} % for braces
\usetikzlibrary{fit} % for fit=(A) (B) (C)...
\usetikzlibrary{positioning} % for the "below of=1cm" type of options
\usetikzlibrary{intersections} % e.g. for pyramid
\usetikzlibrary{calc} % for computing node coordinates from others
\usetikzlibrary{math} % for pgf math stuff
\usetikzlibrary{overlay-beamer-styles} % when \pause glitches in tikz beamer
\usetikzlibrary{arrows} % arrows
\usetikzlibrary{external} % only compile the figure the first time
\tikzexternalize[only named=true, prefix=tikz-cache/]
\immediate\write18{mkdir -p latex-build/tikz-cache}

% used to pass "scale=XX" as includetikz option
\pgfkeys{
  /tikzoptions/.is family, % Namespace
  /tikzoptions/.cd, % Namespace
  scale/.store in=\scale, % Store the value of 'scale'
}

% oplus node, usage \node[oplus=1cm] or any radius value
\tikzset{
  oplus/.style={
    circle, draw,
    minimum size=2*#1, % diameter = 2 * radius
    inner sep=0pt,
    transform shape=true,
    append after command={
      % Draw vertical line
      \pgfextra{
        \def\r{#1}
        % Draw vertical line
        \draw ($(\tikzlastnode) + (0,-\r)$) --
        ($(\tikzlastnode) + (0,\r)$);
        % Draw horizontal line
        \draw ($(\tikzlastnode) + (-\r,0)$) --
        ($(\tikzlastnode) + (\r,0)$);
      }
    },
  }
}

% usage: \includetikz[scale=XX]{fig}, fetches from my local database
\newcommand{\includetikz}[2][scale=1]{
  \bgroup
  \pgfkeys{/tikzoptions,#1} % currently only accepts 'scale' (april 9 2025)
  \tikzsetnextfilename{#2}
  \include{tex-macros/tikz-figures/#2}% Include the standalone TikZ file
  \egroup
}

% usage: \inputtikz[scale=XX]{fig}, fetches from my local database
\newcommand{\inputtikz}[2][scale=1]{
  \bgroup
  \pgfkeys{/tikzoptions,#1} % currently only accepts 'scale' (april 9 2025)
  \tikzsetnextfilename{#2}
  \input{tex-macros/tikz-figures/#2}% Include the standalone TikZ file
  \egroup
}


\newcommand{\bigOt}{\widetilde{\mathcal{O}}}
\newcommand{\bigO}{\mathcal{O}}

\colorlet{my-purple}{solarized-violet}
\colorlet{my-red}{solarized-red}
\colorlet{my-orange}{solarized-orange}
\definecolor{my-darkor1}{HTML}{96240f}
\definecolor{my-darkor2}{HTML}{7f180c}
\definecolor{my-lb1}{HTML}{00557e}
\definecolor{my-lb2}{HTML}{013157}
\colorlet{my-blue}{solarized-blue}
\definecolor{my-bl1}{HTML}{0d08a9}
\definecolor{my-bl2}{HTML}{04037f}
\colorlet{my-cyan}{solarized-cyan}
\colorlet{my-green}{solarized-green}
\definecolor{my-gr1}{HTML}{097e3e}
\definecolor{my-gr2}{HTML}{025520}
\colorlet{my-yellow}{solarized-yellow}
\colorlet{my-magenta}{solarized-magenta}

\hypersetup{
  colorlinks = true, % fixes ugly boxes
  linkcolor = my-red,
  citecolor = my-green,
  filecolor = my-cyan,
  urlcolor = my-magenta,
}
\setcounter{tocdepth}{2} % for metadata: 0 = to title, 3 = to subsubsecs
\newcommand{\cmark}{\textcolor{my-gr1}{\ding{51}}}
\newcommand{\xmark}{\textcolor{my-red}{\ding{55}}}

% new commands: \bibalias and \acite
% 1. add \bibalias{<alias>}{<source>} in preamble
% 2. \acite{<alias>} expands to \cite{<source>}
% (having different keys for the same citation is not supported by bibtex)
% improved version: takes option \acite[opt]{alias} and multiple aliases
% as in \acite{ref1, ref2}
\makeatletter
\newcommand\bibalias[2]{%
  \@namedef{bibali@#1}{#2}%
}
\newcommand{\acite}{\@ifnextchar[{\acite@opt}{\acite@noopt}}
  \newcommand{\acite@opt}[2][]{%
    \acite@parse{#2}%
    \cite[#1]{\biba@all}%
  }
  \newcommand{\acite@noopt}[1]{%
    \acite@parse{#1}%
    \cite{\biba@all}%
  }
  \newcommand{\acite@parse}[1]{%
    % Remove spaces after commas
    \edef\biba@temp{\zap@space#1 \@empty}%
    \def\biba@comma{}%
    \def\biba@all{}%
    \@for\biba@one:=\biba@temp\do{%
      \@ifundefined{bibali@\biba@one}{%
        \edef\biba@all{\biba@all\biba@comma\biba@one}%
      }{%
        \PackageInfo{bibalias}{%
          Replacing citation `\biba@one' with `\@nameuse{bibali@\biba@one}'
        }%
        \edef\biba@all{\biba@all\biba@comma\@nameuse{bibali@\biba@one}}%
      }%
      \def\biba@comma{,}%
    }%
  }
  \makeatother

  \newcommand{\myorcid}{0009-0003-0016-2667}
  \newcommand{\todoAU}[2][]{\todo[color=my-blue!30,#1]{Aurélien: #2}}
