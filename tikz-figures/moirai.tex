%! TeX root = moirai.tex
\documentclass{standalone}
\usepackage{tikz} % tikz
\usepackage{standalone} % tikz figures are often in standalone
\usepackage{amsfonts} % mathbb etc
\usepackage{amsmath} % crucial package
\usepackage{amsthm} % ams-style theorems
\usepackage{bm} % bold math symbols
\usepackage{xcolor} % more colors
\usepackage[T1]{fontenc} % modern encoding, better than OT1, the default

\newtheorem{definition-fr}{Définition}
\newtheorem{example-fr}{Exemple}
\newtheorem{theorem-fr}{Théorème}
\newtheorem{proposition-fr}{Proposition}
\newtheorem{lemma-fr}{Lemme}
\newtheorem{remark-fr}{Remarque}
\newtheorem{definition}{Definition}
\newtheorem{lemma}{Lemma}
\newtheorem{proposition}{Proposition}
\newtheorem{remark}{Remark}

\usetikzlibrary{decorations.pathmorphing} % for snake, zigzag...
\usetikzlibrary{positioning} % for the "below of=1cm" type of options
\usetikzlibrary{intersections} % e.g. for pyramid
\usetikzlibrary{calc} % for computing node coordinates from others
\usetikzlibrary{overlay-beamer-styles} % when \pause glitches in tikz beamer
\usetikzlibrary{arrows} % arrows
\usetikzlibrary{external} % only compile the figure the first time
\tikzexternalize[only named=true, prefix=tikz-cache/]
\immediate\write18{mkdir -p latex-build/tikz-cache}

% used to pass "scale=XX" as includetikz option
\pgfkeys{
  /tikzoptions/.is family, % Namespace
  /tikzoptions/.cd, % Namespace
  scale/.store in=\scale, % Store the value of 'scale'
}

% usage: \includetikz[scale=XX]{fig}, fetches from my local database
\newcommand{\includetikz}[2][scale=1]{
  \bgroup
  \pgfkeys{/tikzoptions,#1} % currently only accepts 'scale' (april 9 2025)
  \tikzsetnextfilename{#2}
  \include{tex-macros/tikz-figures/#2}% Include the standalone TikZ file
  \egroup
}

% usage: \inputtikz[scale=XX]{fig}, fetches from my local database
\newcommand{\inputtikz}[2][scale=1]{
  \bgroup
  \pgfkeys{/tikzoptions,#1} % currently only accepts 'scale' (april 9 2025)
  \tikzsetnextfilename{#2}
  \input{tex-macros/tikz-figures/#2}% Include the standalone TikZ file
  \egroup
}


\newcommand{\bigOt}{\widetilde{\mathcal{O}}}
\newcommand{\bigO}{\mathcal{O}}

\definecolor{mypurp}{HTML}{8f0c97}
\definecolor{myred}{HTML}{9f103b}
\definecolor{myor}{HTML}{ae3011}
\definecolor{mydarkor1}{HTML}{96240f}
\definecolor{mydarkor2}{HTML}{7f180c}
\definecolor{mylb1}{HTML}{00557e}
\definecolor{mylb2}{HTML}{013157}
\definecolor{mybl1}{HTML}{0d08a9}
\definecolor{mybl2}{HTML}{04037f}
\definecolor{mygr1}{HTML}{097e3e}
\definecolor{mygr2}{HTML}{025520}
\definecolor{myyellow}{HTML}{FFD700}



\makeatletter
\@ifundefined{scale}{
  \def\scale{1}
}
\makeatother


\begin{document}
\newcommand{\tikzS}{$S$}
\begin{tikzpicture}
  \def\arrowlength{1em}
  \def\midgap{2em}
  \def\branchwidth{4em}
  \def\sboxwidth{2.3em}
  \def\oplusr{0.4em}
  \def\linh{1.5em}
  \foreach \i in {0,...,3} {
    \coordinate (x1\i) at ($(\i*\branchwidth,0)$);
    \coordinate (x2\i) at ($(x1\i) + (0, -\arrowlength)$);
    \coordinate (x3\i) at ($(x2\i) + (0, -\linh)$);
    \coordinate (x4\i) at ($(x3\i) + (0, -\midgap)$);
    \coordinate (x5\i) at ($(x4\i) + (0, -\midgap)$);
    \coordinate (x6\i) at ($(x5\i) + (0, -\linh)$);
    \coordinate (x7\i) at ($(x6\i) + (0, -\arrowlength)$);
    \draw[->] (x1\i) -- (x2\i);
    \draw[->] (x6\i) -- (x7\i);
  }
  \draw[->] (x30) -- (x50);
  \draw[->] (x32) -- (x52);
  \coordinate[oplus=\oplusr] (xor1) at (x41);
  \coordinate[oplus=\oplusr] (xor2) at (x43);
  \draw[->] (x31) -- (xor1.north);
  \draw[->] (x33) -- (xor2.north);
  \draw[->] (xor1.south) -- (x51);
  \draw[->] (xor2.south) -- (x53);

  \coordinate (a1) at ($(x20) + (-\sboxwidth, 0)$);
  \coordinate (a2) at ($(x33) + (\sboxwidth, 0)$);
  \draw[rounded corners] (a1) rectangle node {$\mathcal{A}_0$} (a2);

  \coordinate (s10) at ($(x40) + ({(\branchwidth - \sboxwidth)/2}, 0)$);
  \coordinate (s20) at ($(s10) + (\sboxwidth, 0)$);
  \draw[->] (x40) -- (s10);
  \draw[->] (s20) -- (xor1.west);
  \coordinate (sc1) at ($(s10) + (0, {(\sboxwidth)/2})$);
  \coordinate (sc2) at ($(s20) + (0, {-(\sboxwidth)/2})$);
  \draw[rounded corners] (sc1) rectangle node {\tikzS} (sc2);

  \coordinate (s12) at ($(x42) + ({(\branchwidth - \sboxwidth)/2}, 0)$);
  \coordinate (s22) at ($(s12) + (\sboxwidth, 0)$);
  \draw[->] (x42) -- (s12);
  \draw[->] (s22) -- (xor2.west);
  \coordinate (sd1) at ($(s12) + (0, {(\sboxwidth)/2})$);
  \coordinate (sd2) at ($(s22) + (0, {-(\sboxwidth)/2})$);
  \draw[rounded corners] (sd1) rectangle node {\tikzS} (sd2);

  \coordinate (b1) at ($(x50) + (-\sboxwidth, 0)$);
  \coordinate (b2) at ($(x63) + (\sboxwidth, 0)$);
  \draw[rounded corners] (b1) rectangle node {$\mathcal{A}_{1}$} (b2);

\end{tikzpicture}
\end{document}
